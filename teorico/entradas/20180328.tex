\chapter{Amplificadores Realimentados}
\textbf{AGREGAR ALGO DE LA PRIMERA CLASE}

\section{Tipos de amplificadores}
\textbf{SACAR LAS SUBSECCIONES Y USAR FIGURAS}
\subsection{Amplificador de Tension}
\textbf{HACER CIRCUITO DE AMPLI DE TENSION}

\subsection{Amplificador de Corriente}
\textbf{HACER CIRCUITO DE AMPLI DE CORRIENTE}

\subsection{Amplificador de Transconductancia}
\textbf{HACER CIRCUITO DE AMPLI DE TRANSCONDUCTANCIA}

\subsection{Amplificador de Transresistencia}
\textbf{HACER CIRCUITO DE AMPLI DE TRANSRESISTENCIA}

\subsection{Comparacion de amplificadores}
\textbf{HACER EL CUADRO COMPARATIVO}


\section{El amplificador realimentado}
\textbf{NO TE OLVIDES DE HACER EL DIAGRAMA EN BLOQUES}


\section{Ganancia y Desensibilidad}
\textbf{EL GRAFIQUITO Y LAS ECUACIONES}


\section{Tipos de Realimentacion}
\textbf{CUADRO COMPARATIVO DE REALIMENTACION}
\textbf{EL OTRO CUADRO COMPARATIVO}


\section{Impedancias de entrada y salida en realimentados}
\textbf{CIRCUITO DEL AMPLI}
\textbf{DESARROLLO DE LAS ECUACIONES}
