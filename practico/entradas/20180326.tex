\chapter{TP 1 Realimentacion}

\section{Introduccion}
\textbf{ESQUEMATICO}
\textbf{ECUACIONES}

Objetivos de la realimentacion negativa:
\begin{itemize}
	\item Aumentar la respuesta en frecuencia
	\item Aumentar la estabilidad (Temperatura, envejecimiento 
					y cambio de componentes)
\end{itemize}
Desventajas: 
\begin{itemize}
	\item Disminuye la ganancia
\end{itemize}

\textbf{SENSIBILIDAD}

\section{Topologia de realimentacion}
\textbf{LOS CUATRO ESQUEMATICOS COMO FIGURAS, CON SUS ECUACIONES CORRESPONDIENTES}

\section{Ejercicio de calculo}
\textbf{ESQUEMATICO DE LA CONSIGNA}
\textbf{CIRCUITO EQUIVALENTE}
\textbf{DESARROLLAR EL EJERCICIO}
